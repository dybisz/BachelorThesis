\documentclass{article}

\usepackage{pdfpages}
\usepackage{graphicx}
%\usepackage[export]{adjustbox}
%\usepackage{tabu}
\usepackage{xcolor,colortbl}

\begin{document}

\vspace*{3ex}
\begin{flushright}
{\large 3 March 2015}
\end{flushright}

\begin{flushleft}
{\large Jakub Ciecierski\\

}
\end{flushleft}

\hskip3cm

\begin{center}

\Large {\bf
	Cellular automaton
}

\Large {\bf 
	Requirement specification 
}

\vskip2ex

\vspace{60pt}
\includegraphics[width=80mm]{images/mini.PNG} \\
\end{center}

\vskip20ex

\newpage
\tableofcontents
\newpage
\section{Schedule} \par

\begin{center}

	\begin{tabular}{| l | p{7cm} |}

		\hline
		\cellcolor[HTML]{C0C0C0}Date & \cellcolor[HTML]{C0C0C0} Asset \\
		
		\hline
	  	2015-04-02 & Technical project \\
		
		\hline
	  	2015-04-23 & Code of modules\\
	
		\hline
		2015-04-30 & version 0.98\\

		\hline
		2015-05-07 & version 0.99\\	
		
		\hline
		2015-05-14 & version 1.00\\
		
		\hline
		2015-05-28 & Test report\\
		
		\hline
		2015-06-11 & Acceptation \\
		
		\hline
	
	\end{tabular}

\end{center}

\section{Document metric}


\begin{table}[h]
\hspace*{-3.1cm}
\large
\begin{tabular}{|
>{\columncolor[HTML]{C0C0C0}}l |l|l|l|l|l|}
\hline
\multicolumn{6}{|l|}{\cellcolor[HTML]{C0C0C0}Document metric}                                                                                                                                         \\ \hline
Project:       & Cellular Automaton                                                       & \cellcolor[HTML]{C0C0C0}Company: & \multicolumn{3}{l|}{WUT}                                               \\ \hline
Name:          & \multicolumn{5}{l|}{Requirement specification}                                                                                                                                       \\ \hline
Topics:        & \multicolumn{5}{l|}{Business analysis of the product}                                                                                                                                       \\ \hline
Author:        & \multicolumn{5}{l|}{Jakub Ciecierski}                                                                                                                                                \\ \hline
File:          & \multicolumn{5}{l|}{requirement\_specification.pdf}                                                                                                                                      \\ \hline
Version no:    & 0.1                                                                      & \cellcolor[HTML]{C0C0C0}Status:  & Under development & \cellcolor[HTML]{C0C0C0}Opening date: & 2015-03-03 \\ \hline
Summary:       & \multicolumn{5}{l|}{Business analysis of application that allows for creating a cellular automaton}                                                                                                           \\ \hline
Authorized by: & \begin{tabular}[c]{@{}l@{}}Władysław Homenda\\ Lucjan Stapp\end{tabular} & \multicolumn{3}{l|}{\cellcolor[HTML]{C0C0C0}Last modification date:}                         & 2015-03-03 \\ \hline
\end{tabular}
\end{table}



\section{History of changes}

\begin{table}[h]
\hspace*{-2.1cm}
\large
\begin{tabular}{|l|l|l|l|}
\hline
\multicolumn{4}{|l|}{\cellcolor[HTML]{C0C0C0}History of Changes} \\ \hline
Version         & Date         & Who        & Description        \\ \hline
0.1         & 2015-03-03         & Jakub Ciecierski        & Definition of the main purpose of the document       \\ \hline
\end{tabular}
\end{table}


\section{Glossary} 

\newpage
\section{Goal}
The main goal of this project is to deploy application, which will create an automaton for given input data. Produced automaton will be a an accurate classifier of objects represented
by the input data. The program is dedicated to reasearch laboratory, hence it is lumbered with the following assuptions.


First of all, all users will be scientists, so precision of calculations and reliability is vital. We want to be sure about results given by the application to such an extent, that they will be publishable. It is also carrying need for specific format of the output - by default latex tables and .xls files. Similarly input is in form of .xls files.


Next thing that we want to stress out is platform and design. As for target system, linux is unquestionable choice. All work stations are running Arch Linux and we want the program to be operable on all of them. Although most of the researchers work inside the laboratory, some of us are using SSH protocol to communicate. This causes the need for plain console application - configurable using flags or simple question/answer scheme. 

But we do not want to limit ourselves only to this approach - finally vast majority of us use computers via the standard X Window System and want to benefit from it. For those who does, we want to present simple GUI based program to configure, run and monitor the process of calculating automata. It will have all functionalites of console part, but will be more easy on the eye and simpler to use for non computer scientist.


Last but not least, we will tackle resources consuption and critical situation handling. On this point let us be clear: we want accurate results - neither time nor memory are important. The assumtion of course holds to some reasonable extent - we do not want to wait a month for program's output, but we are rather used to wait for couple of days. Great solution in this case would be ability to adjust complexity of calculations and, what follows, time needed to complete. With such an estimation, we could easily schedule our work.

\newpage
\section{User stories}

%
% GUI User Stories
%
\subsubsection{GUI}
As a user I want to:
\begin{itemize}
	\item
		load data using file explorer.
	\item
		load data using drag and drop procedure.
	\item
		adjust computation precision and see estimated time to complete.
	\item
		select output format as .xls file.
	\item
		select output format as latex table.
	\item
		select destination folder of the output
	\item
		decide if test should be rerun in case of failure/interruption.
	\item 
		start computation for loaded data.
	\item
		stop specific computation.
	\item
		stop all computations.
	\item 
		monitor number of currently running computations
	\item
		monitor estimated time of all computations
	\item
		monitor progress of single computation.
	\item
		close application at any time
\end{itemize}


	

\newpage

\section{Functional Requirements}
Priority
\begin{itemize} 
\item 1 - must be implemented
\item 2 - can be implemented optionally 
\item 3 - is a nice addition, but not needed.
\end{itemize} 

\begin{center}
\hspace*{-2.1cm}
	\begin{tabular}{| l | p{7cm} | p{5cm} | l |}
	
		\hline
	  	ID & Requirement & Comments & Priority \\
		\hline
		
		1 & 
		The system provides a Grid options
		allowing for changing the size, colour of cells and
		enabling/disabling wrapping option&
		The colour of cells represent a state of the cell.
		In other words the user can choose in what state to put a cell into. &
		1 
		\\ \hline

		1.1 & 
		The system should allow grid maneuvers, zooming in/out and if the
		entire grid is not visible in one screen, possibility of moving around the grid&
		 &
		1 
		\\ \hline

		2 & 
		The system provides a Rule editor in which
		the user can create, edit and save rules. & 
		 &
		1 
		\\ \hline
		
		2.1 & 
		By clicking create rule button in Rule editor, the application will open a fresh rule
		creation window & 
		 &
		1 
		\\ \hline

		2.2 & 
		By clicking load button in Rule editor, the application will open a browser which
		will allow the user to find saved rules & 
		 &
		1 
		\\ \hline

		2.3 & 
		By clicking save button in Rule editor, the application will make sure that name for the
		rule is provided and then will save the rule in specified by the user location & 
		 &
		1 

		\\ \hline


	  \hline
	\end{tabular}
\hspace*{-2.1cm}
	\begin{tabular}{| l | p{7cm} | p{5cm} | l |}
		\hline
	  	ID & Requirement & Comments & Priority \\
		\hline

		2.4 & 
		The system provides three different neighborhood environments
		in which the user can create rules, 4-point, 8-point, 24-point  & 
		See Glossary / Neighborhood for more information &
		1 
		\\ \hline
		

		2.5 & 
		The application provides special file extension for saving and keeping rules & 
		 &
		1 
		\\ \hline
				
		2.6 &
		For 4-point and 8-point environments the system should provide a way to create rules
		in which positions of neighbors relative to the cell are considered. If
		a transition is not defined then this transition does not change the state of current cell
		 &
		The user can choose to what state current cell transitions, based on this cell's state 
		and states of his neighbors &
		1 
		\\ \hline

		2.7 &
		For 24-point environment system should provide a may of creating rules in which 
		the user specifies number of neighbors in each column
		&	
		This environment can be represented as a 5 by 5 matrix with the current cell in the middle &
		1 
		\\ \hline

		2.8 &
		For 4-point, 8-point and 24-point environments the system should provide a simplified
		mode of creating rules in which the user inputs only number of neighbors in given state
		in the neighborhood for a current cell state. &
		The user inputs number of neighbors with given state which should appear for the
		cell to transition to another specified state &
		2 
		\\ \hline
				
		3.1 & 
		The system provides a step-by-step button which computes next generation & 
		 &
		1 

		\\ \hline



	\end{tabular}
\hspace*{-2.1cm}
	\begin{tabular}{| l | p{7cm} | p{5cm} | l |}
		\hline
	  	ID & Requirement & Comments & Priority \\
		 \hline

		3.2 & 
		The system provides next-N button which computes next N generations,
		the N must be easily chosen by the user & 
	 		&
		1
		\\ \hline

		3.3 & 
		The system provides a run button which will
		start the animation of consecutive generations & 
		 &
		1
		\\ \hline
		
		3.4 & 
		The system provides way to change speed of which the animation is drawn in the 'run' mode & 
		 &
		1
		\\ \hline

		4.1 & 
		The application allow user to draw cells on the grid & 
		 &
		1
		\\ \hline		

		4.2 & 
		The application provides a way for user to save grid state into patterns, additionally
		the pattern can have a rule attached to it, which later can be loaded into the grid.  & 
		A pattern editor view component should be created.
		The grid state consists of its size, states of cells and other grid options.&
		2
		\\ \hline				

		4.3 & 
		The application provides Browser window in which the 
		user can browse saved rules and patterns & 
		 &
		1 
		\\ \hline
	
		4.4 & 
		The application allows the user to have multiple grids opened. &
		 &
		2 
		\\ \hline

		4.5 & 
		The application should have example of simple game of Life called
		Conway's Game of Life &
		 &
		2 
		\\ \hline
	
	\end{tabular}

\end{center}

\newpage

\section{Non Functional Requirements}


\end{document}
